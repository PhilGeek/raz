%!TEX TS-program = pdflatex 

\documentclass[a4paper,12pt]{article}

%Packages
\usepackage{palatino}
\usepackage{setspace}
% \doublespace

% Bibliography
\usepackage[round]{natbib} 

% Title Information
\title{Respecting Value} %\thanks{Thanks to MGF Martin, V\'eronique Munoz-Dard\'e, Joseph Raz, and Charles Travis for their help and encouragement. Thanks as well to two anonymous referees whose comments led to substantial improvements.}
\author{Mark Eli Kalderon}

%%% BEGIN DOCUMENT
\begin{document}

% Title Page
\maketitle

% Main Content

\section{Introduction} \label{sec:introduction} % (fold)

In chapter four of \emph{Value, Respect, and Attachment}, \cite{Raz:2001ps} describes his views about respect as having a Kantian origin. Raz introduces his discussion by raising three questions concerning Kant's account: 
\begin{itemize}
	\item Why is there no analogue of the feeling of respect in Kant's treatment of theoretical reason? (\S\ref{sec:why_is_there_no_analogue_of_respect_in_theoretical_reason}) 
	\item What is the proper object of respect---the moral law or people considered as ends in themselves? (\S\ref{sec:what_is_the_object_of_respect}) 
	\item How could people be ends, let alone ends in themselves, if ends are states of affairs intentionally brought about by action? (\S\ref{sec:how_could_people_be_ends}) 
\end{itemize}
These questions form the basis of Raz's criticisms of Kant, criticisms which, in turn, motivate Raz's own account of respect. Unfortunately, I do not believe that Raz's criticisms can be sustained in the end. Kant's doctrines, properly interpreted, evades these criticisms. 

The present essay thus has historical aims, but not only these---there are  philosophical aims as well. 

Part of the point of getting Kant right is to avail ourselves of the philosophical insights that Kant affords---insights that we ignore at our peril in understanding the moral phenomenon of respect. In this way, history contributes to philosophy. Not only does a proper interpretation of Kant afford us with genuine philosophical insights, but philosophy, in turn, can contribute to history. Some of Raz's criticisms crucially turn on misinterpretations of Kant's views---misinterpretations, in turn, facilitated by \emph{philosophical} errors. Very often a correct philosophy will see us through to a proper interpretation of Kant. In addition, as we will see, Raz's own insights into the metaphysics of value provide an elaboration and supplementation of Kant's derivation of the Formula of Humanity---just as Kant's insights into the intuition of human dignity provide an elaboration and supplementation of Raz's account.

Thus I will be pursuing both history and philosophy with an eye to illuminating each.

Raz's three questions will be considered, in turn, in the following three sections. In his discussion of Kant, Raz sets aside the phenomenal--noumenal distinction, and I will follow him in this. Some of the limitations of this procedure will be discussed in the final section.


\section{Why Is There No Analogue of Respect in Theoretical Reason?} \label{sec:why_is_there_no_analogue_of_respect_in_theoretical_reason} % (fold)

Raz claims that there is a puzzling asymmetry between Kant's treatment of theoretical and practical reason: 
\begin{quote}
	\ldots\ Kant treats practical reason differently from theoretical reason. There is no analogue to respect in his treatment of theoretical reason. But if reason can determine the will to believe one thing or another, with no special feeling of respect being invoked, why can it not determine the will to will one thing or another without such feeling? \citep[132--3]{Raz:2001ps} 
\end{quote}

One asymmetry that Kant postulates between theoretical and practical reason is the different philosophical challenges they pose: 
\begin{quote}
    Yet we cannot consider without admiration how great an advantage the practical faculty of appraising has over the theoretical in common human understanding. In the latter, if common reason ventures to depart from the laws of experience and perceptions of the senses it falls into sheer incomprehensibilities and self-contradictions, at least into a chaos of uncertainty, obscurity, and instability. But in practical matters, it is just when common understanding excludes all sensible incentives from practical laws that is faculty of appraising first begins to show itself to advantage. (\textsc{g} 4:404)
\end{quote}
In the theoretical case, Kant argues that pure reason has a tendency to exceed its proper limits and thus involve itself with the antinomies. In contrast, in the case of practical reason, it is \emph{empirical} and not \emph{pure} practical reason, at the prompting of inclination, that tends to exceed proper limits, especially when the moral law and its basis in the nature of persons is unclear to us. Perhaps, then, the focus on the feeling of respect in Kant's account of practical reasoning is a reflection of the moral and philosophical challenges posed by our empirical natures shaped as they are by their social setting. 

Inclination is that aspect of our empirical nature that is the source of these moral and philosophical challenges (challenges whose nature will be discussed in detail in the next section.) Of course, as \citet[133]{Raz:2001ps} observes, inclination can distort theoretical judgment as well. Nor is Kant blind to this. However, while self-love, wishful thinking, vanity, etc.\ may lead to error, they do not invariably lead to `sheer incomprehenisibilities and self-contradictions' the way the natural temptation to exceed reason's proper limits does. In the practical domain, in contrast, the operation of inclination is more disastrous, being the root of a misconception of our value and the source of moral wrongdoing.

To get a better sense of the asymmetry, let's consider the nature of inclination. In a note in the \emph{Groundwork}, Kant defines inclination as `the dependence of the faculty of desire on feeling' (\textsc{g} 4:414n). That is not terribly helpful. This is one example of the way in which the underlying moral psychology of the \emph{Groundwork} can be frustratingly inexplicit. The moral psychology that underlies Kant's ethics will be refined, elaborated, and, in some respects, revised in subsequent work. However, a subsequent elaboration provides the basis of an interpretation of the, otherwise puzzling, characterization of inclination as `the dependence of the faculty of desire on feeling'.

In the \emph{Critique of the Power of Judgement}, Kant divides the mind's powers into three faculties (\textsc{ku} 5:198)---\emph{the faculty of cognition}, \emph{the faculty of feeling}, and \emph{the faculty of desire}. The faculty of desire is the mind's capacity to produce an object by means of a representation. It stands midway between the faculties of cognition and feeling in that it depends on each: Desiring an object involves having a \emph{representation} of an object accompanied by a \emph{feeling} of pleasure. In empirical desire, the feeling of pleasure is the contingent effect of the representation of an object impinging on one's faculty of feeling. When an empirical desire becomes habitual, it is an inclination. It is the way in which the pleasure involved in the habitual empirical desire depends on the contingent configuration of the faculty of feeling (which can vary from one finite rational being to another) that Kant plausibly has in mind when he defines inclination as the `dependence of the faculty of desire on feeling'.

Following Hume, Kant holds that the regular operation of habit occurs unreflectively. However, when a person reflects and judges that pleasure is regularly connected with an object, a person takes an interest in that object. Thus awareness of an inclination produces an interest in the object (\textsc{g} 4:413). When an interest arises from inclination it is a pathological interest. Interest, however, can arise from rational grounds as well. When it does, it is a practical interest.

Like Hume, Kant claims that all action involves desire. Unlike Hume, Kant claims that not all desire is empirical. (Though perhaps this Kantian take on Hume is overly narrow. See Wiggins, \citeyear{Wiggins:1995fk}, for an account of just how much a Humean philosophy can accomodate Kantian insight.) Even in circumstances where the objects of inclination are grounds for action, it is reason that determines whether this is so. Rational beings view their empirical desires as potential grounds of rational choice. Pure desires, in contrast, are not grounds of rational choice, even potentially, but are noncognitive responses to these grounds. Thus a person acts from duty, not because he antecedently wants to; rather, a person desires to do his duty because he recognizes the reasons he must. The interest in the moral law involved in acting from duty is practical, not pathological. The desire in acting from duty is pure, and the feeling involved in this pure desire is the feeling of respect.

Kant's talk of the feeling of respect can mislead. Respect is not, or not merely, a feeling. It is also a way of treating people (as ends in themselves and never merely as means). The feeling of respect is part of the noncognitive response to the reasons there are for such treatment.

The asymmetry that Raz observes between theoretical and practical reason is the consequence of: 
\begin{itemize}
	\item Kant's conception of desire as a representation of its object accompanied by a feeling of pleasure, 
	\item Kant's subscription to the Humean claim that all action proceeds from desire, and 
	\item Kant's claim the determination of judgment does not involve desire in the way that the determination of the will does. 
\end{itemize}
One might \emph{disagree} with the postulation of this asymmetry, but it is not particularly \emph{puzzling}.

An important observation about pure desire forestalls one potential source of disagreement. Kant understands the Humean claim that all action proceeds from desire as a \emph{psychological} and not a \emph{normative} claim. It is a claim about the psychology of practical deliberation and not a claim about what reasons there are. Pure desires are noncognitive responses to a person's reasons as he understands them to be. They are a response to what reasons there are and not themselves reasons for action. So doubts about the extent to which desires could be reasons for action are no embarrassment for at least this aspect of Kantian moral psychology \citep[for such doubts see, inter alia,][]{Scanlon:1998hb, Raz:2000tm}. 

If pure desires are not sources of reasons then why are they psychologically necessary? In \S\ref{sub:on_not_treating_people_merely_as_a_means} we will see that pure desires play an important explanatory role in the motivational psychology of a virtuous sensibility. As we will see, this is morally, as well as psychologically, significant. Indeed, it is an important part of the resolution of the moral and philosophical challenges posed by our empirical natures.

% section why_is_there_no_analogue_of_respect_in_theoretical_reason (end)

\section{What Is the Object of Respect?} \label{sec:what_is_the_object_of_respect} % (fold)

\citet{Raz:2001ps,Raz:2002vn} distinguishes between \emph{independent} and \emph{dependent} conceptions of respect. 

Any moral doctrine of respect essentially involves an independent conception of respect. If there are duties to respect people, then there are special reasons to respect people, to treat them in ways simply because they are to be respected. Treating them in these ways is what respect requires that we do. 

A moral doctrine of respect is opposed by those who conceive respect to be dependent. If respect is dependent, then there are no special reasons to respect people, to treat them in ways simply because they are to be respected. Rather, respect for people is merely a byproduct of acting from sound independent reasons. Such reasons are independent in the sense that the force of their recommendation does not depend on the requirements of respect. While \citet{Raz:1986lq} once endorsed the byproduct view, \citet[127]{Raz:2001ps} now maintains that `we should only accept it if a more substantive understanding of reasons for respect proves elusive'. 

In a long note in section one of the \emph{Groundwork} Kant claims that the fundamental object of respect is the moral law and that persons are only objects of respect derivatively, as exemplars of the law (\textsc{g} 4:401n). On this basis, \citet[131]{Raz:2001ps} notes the irony of contemporary defenders of a moral doctrine of respect citing Kant as a predecessor:
\begin{quote}
	The moral law, rather than people, is the object of respect. And there are no duties or requirements to respect anyone or anything. Rather, (a feeling of) respect arises in us as the moral law determines our will. Contemporary discussions of respect for persons strive to articulate and defend a particular moral principle or doctrine, that is, the one saying, in brief, that we must respect people. That principle has little to do with Kant's doctrine of respect, which, to repeat, does not add to the content of morality, but says that whenever we perform an action because we rationally believe to is our moral duty to do so we act out of respect for the moral law. \citep[134]{Raz:2001ps}
\end{quote}
So, whereas contemporary defenders of the moral doctrine of respect endorse an independent conception of respect, according to Raz, Kant, from whom they derive inspiration, endorses, instead a dependent conception of respect.

Raz explains the influence of Kant's moral philosophy on contemporary discussions of personal moral respect in terms of the fact that Kant does not always adhere to his official doctrine, particularly, when he claims that persons are ends in themselves and must always be treated as such and never merely as a means: 
\begin{quote}
	The two grounds of respect (i.e., people exemplifying the moral law by following it, and people being ends in themselves) are quite different and could lead to different reasons for action. However, for Kant the two converge, and so the slippage in meaning is easy to overlook. \citep[136]{Raz:2001ps} 
\end{quote}
That people are ends in themselves and must be treated as such and never merely as means seems to posit a special reason to respect people and so encapsulates an independent conception of respect \citep[for discussion relevant to the distinction Raz is drawing see][]{Darwall:1977ty,Frankena:1986sf}.

The two grounds of respect are conceptually independent, so there is some justice to his worry that they might ground independent reasons for action. Whether they do \emph{as Kant understands them}, however, depends on details of their role in Kant's system. I must confess, however, that I find the charge of a slippage in meaning unfair. Perhaps Kant is proposing a substantive identification. Perhaps respect for the moral law \emph{just is} respect for persons considered as ends in themselves \citep[see][for a similar suggestion]{Velleman:2006nx}. 

Before resolving this issue, let's first complicate it by considering Kant's controversial views about the illegitimacy of interpersonal moral comparisons \citep[see][ch.~4, \S6]{Wood:1999zy}.

Kant writes: 
\begin{quote}
	Fontenelle says: `I bow to a great man, but my mind does not bow.' I can add: before a humble plain man, in which I perceive an uprightness of character in greater measure than I am conscious of in myself, \emph{my mind bows} whether I choose or not, however high I may carry my head in order that he may not forget my superiority. Why? His example holds a law before me which strikes down my self conceit when I compare my own conduct with it. (\textsc{kpv} 5:76-77) 
\end{quote}

On a casual reading, this passage is an expression of humility given a comparison between Kant and the humble plain man. However, Kant is making no such comparison. When Kant acknowledges the uprightness of the humble plain man, he recognizes the emptiness of his sense of superiority and experiences humility when he compares his conduct with the moral law. Kant is reinterpreting moral judgments we make comparing ourselves with others as judgments comparing ourselves with the moral law.

This can seem like a theoretical commitment taken to the extreme. Kant's doctrine that the object of respect is the moral law leads him to reinterpret interpersonal moral comparisons familiar from everyday life. Kant is indeed engaged in a revision of, if not the moral judgments we make, then at least our understanding of them. But there is a distinctively moral, and not merely theoretical, point to this revision---a moral point that makes it plain that Kant is indeed making the substantive identification that I claim he is.

The moral point has positive and negative aspects.

The positive aspect concerns the absolute dignity of humanity. Being absolute, the dignity of rational beings is \emph{equal}. Kant takes this consequence very seriously. According to Kant, the serpent spoke truly to Eve, when he claimed that by eating the fruit of the tree of knowledge of good and evil they would become equal to God (\textsc{ma} 8:115). In gaining practical reason (knowledge of good and evil), Adam and Eve become equal to God in the sense that every rational being has equal and absolute worth. The ban on interpersonal moral comparisons is the expression of the dignity of humanity which is absolute and hence equal between rational beings.

The negative aspect concerns a diagnosis of human wrongdoing. Like Hume, Kant takes it for granted that we all more or less know right from wrong. However, they differ in their attitudes towards human nature---Humean cheerfulness is counterbalanced by Kantian suspicion. On Kant's view, we are locked in a struggle between competing conceptions of our value. 

On the one hand, we are rational beings and thus endowed with the capacity of free rational choice. Our free reason enjoins us to show equal respect for the dignity of all humanity and constrains us to act on principles capable of uniting all human ends into a harmonious system. Our free reason is the source of a conception of our value based on the absolute dignity of our common humanity. 

On the other hand, we are finite beings and thus subject to inclination and need. Inclination and need are the source of both our mutual interdependence and mutual hostility. Kant, inspired by Rousseau, describes this as our `unsocial sociability'. This unsocial sociability is the source of a conception of our value based on competition and comparative advantage. 

These conceptions of our value are not only distinct but practically conflict as well. Specifically, our unsocial sociability is the source of the continual temptation to make exceptions for ourselves to universal moral principles to the advantage of inclination. (See \S~\ref{sub:on_not_treating_people_merely_as_a_means} for more discussion of this.) So the ban on interpersonal moral comparisons is due, in part, to the fact that interpersonal comparative judgments are a mode of thinking and evaluation rooted in a misconception of our value that is the source of human wrongdoing.

What seems to us to be a curious focus on the moral law is, by Kant's lights, at once the expression of respect for the dignity of humanity and a condemnation of an alternative and misconceived conception of our value based on competition and comparative advantage. Thus, respect for the moral law \emph{just is} respect for persons considered as ends in themselves.

Indeed, this identification is implicit in Kant's account of his encounter with Rousseau: 
\begin{quote}
	I am an inquirer by inclination. I feel a consuming thirst for knowledge, the unrest which goes with the desire to progress in it, and satisfaction with every advance in it. There was a time when I believed this constituted the honor of humanity, and I despised the people, who know nothing. Rousseau set me right about this. The blinding prejudice disappeared. I learned to honor human beings, and I would find myself more useless than the common laborer if I did not believe that this attitude of mine can give worth to all others in establishing the rights of humanity. (\textsc{ak} 20:44) 
\end{quote}
The first thing to notice is that this account is a \emph{conversion narrative}---Kant was lost but now is found, was blind but now can see. As a conversion narrative, it self-consciously reports an experience that affords a moral insight so profound that it forever transforms Kant's character. The moral insight at the heart of Kant's conversion motivates his later theoretical articulation of it. Prior to his encounter with Rousseau's work, Kant conceived of his self-worth in terms of his comparative intellectual advantage over others. Afterwards, his conception of his value and the value of others is transformed. He would be lower than the common laborer that he formerly despised for ignorance if he did not acknowledge and act on behalf of the equal rights of \emph{all} humanity. Kant's irony here is pointed---he rejects a conception of his value based on comparison and ranking in favor of a value shared unconditionally and without limitation by all. Kant is a \emph{radical} egalitarian---the dignity of humanity is equal not only in the sense that it is shared by all but also in the sense that its value is \emph{incomparable}. Regarding the value of humanity as an object of comparison dishonors humanity---`The opinion of inequality makes people unequal. Only the teaching of M.\ R can bring it about that even the most learned philosopher with his knowledge holds himself, uprightly and without the help of religion, no better than the common human being.' (\textsc{ak} 20:176). Proper respect for humanity requires that one only compare one's conduct with the moral law and not with the conduct of other rational beings. Respect for the moral law just is respect for persons considered as ends in themselves.

One might object as follows: To be sure, the dignity of humanity, being absolute, is equal, in the sense that it is shared unconditionally and without limitation by all rational beings. But the \emph{moral} or \emph{inner} worth of human beings varies from one individual to another---there are saints and sinners among us. If inner worth can so vary why can't this be acknowledged by an explicit comparative judgment?

Comparison of inner worth would only be possible if there were a common standard by which to judge. According to Kant, however, there is no such common standard and so no such comparison is possible.

A person's inner worth is not a matter of their acting in conformity with duty; rather, it is a matter of their acting from duty in the absence of positive inclination or in the face of contrary inclination.  
One might wonder whether this is really Kant's doctrine. After all, Kant, in the \emph{Metaphysics of Morals}, writes:
\begin{quote}
	Ethical duties must not be determined in accordance with the capacity to fulfill the law that is ascribed to human beings; on the contrary, their moral capacity must be estimated by the law which commands categorically and so in accordance with our rational knowledge of what they ought to be \ldots\ not in accordance with the empirical knowledge we have of them as they are. (\textsc{mm}: 6:404)
\end{quote}
There is no contradiction, however. Our duties may not be determined by our empirical capacity to fulfill them, but our moral or inner worth when we act as duty requires is so determined, at least in part. Moral or inner worth crucially depends on our \emph{motivation} in fulfilling our duty. Earlier in the same work Kant writes:
\begin{quote}
	\emph{Subjectively}, the degree to which an action \emph{can be imputed} (\emph{imputabilitas}) has to be assed by the magnitude of the obstacles to be overcome. --- The greater the natural obstacles (of sensibility) and the less the moral obstacle (of duty), so much the more merit is to be accounted for a \emph{good deed}, as when, for example, at considerable self-sacrifice I rescue a complete stranger from great distress. (\textsc{mm} 6:228)
\end{quote}

Notoriously, Kant consistently maintains that human character is ultimately unknowable, at least to its full extent:
\begin{quote}
	For a human being cannot see into the depths of his own heart so as to be quite certain, in even a single action, of the purity of his moral intention and the sincerity of his disposition, even when he has no doubt about the legality of that action. Very often he mistakes his own weakness, which counsels him against the venture of a misdeed, for virtue (which it is concept of strength); and how many people who have lived long and guiltless lives may not be merely fortunate in having escaped so many temptations. In the case of any deed it remains hidden from the agent himself how much pure moral content there has been in his disposition (\textsc{mm}: 6:393)
\end{quote}
So even if there were an approximate standard sufficient for rough and ready comparisons of inner worth, we could not know it. And that that means that there could be no standard \emph{by which we judge}. To be a standard by which we judge, the standard must be epistemically available to us so as to guide our judgment. The moral challenges facing an individual are unknowable, no common standard is available, and hence no comparison of inner worth is possible.

As I have emphasized, Kant's focus on the moral law has positive and negative aspects---it is at once the expression of respect for the dignity of humanity and a condemnation of a misconception of our value that is the source of human wrongdoing. \citet{Velleman:2006nx} also endorses the identification of respect for the moral law with respect for persons as ends in themselves, though he urges this identification on different grounds:
\begin{quote}
	[R]everence for the law, which has struck so many as making Kantian ethics impersonal, is in fact an attitude toward the person, since the law that commands respect is the ideal of a rational will, which lies at the heart of personhood. \citep[81]{Velleman:2006nx}
\end{quote}
Respect for the law understood as `reverence for the intelligible aspect under which our will is an ideal, or law, to its empirical self' \citep[80]{Velleman:2006nx} may capture the attitude owed to the dignity of humanity, but it leaves Kant's negative moral point about the misconception of a person's value in terms of comparative advantage entirely out of account. This is a reflection of a more fundamental difference. Velleman explains the way in which respect for the moral law is a personal attitude in terms of the way in which the moral law figures as an ideal of personhood. In contrast, I explain the way in which respect for the moral law is a personal attitude in terms of the dignity of humanity which, being incomparable, requires that no comparison be made.

Supposing such an identification to be intelligible, what, if anything, does this imply about Raz's distinction between independent and dependent conceptions of respect? Recall, on an independent conception, there are special reasons to respect people, to treat them in ways simply because they are to be respected; whereas on a dependent conception, there are no special reasons to respect people---respect for people is merely a byproduct of acting from sound independent reasons (such reasons are independent in the sense that the force of their recommendation does not depend on the requirements of respect). If respect for the moral law \emph{just is} respect for persons considered as ends in themselves, is respect, so conceived, independent or dependent? Or does the distinction even make sense?

Raz's distinction is sound, and the identification of respect for the moral law with respect for people as ends in themselves, urged on the grounds I have attributed to Kant, is an independent conception of respect. Recall, Raz reasoned that since the moral law is the primary object of respect, respect, so conceived, is dependent. The feeling of respect arises in us as the moral law determines our will, and this can seem like a byproduct view of respect. However, this appearance is misleading. What Raz misses is Kant's reason for maintaining that the moral law is the primary object of respect. According to Kant, we owe it to people, given the incomparable value of their dignity, to respect the moral law by, inter alia, comparing our conduct only with the moral law and not with the conduct of others.

% section what_is_the_object_of_respect (end)

\section{How Could People Be Ends?} \label{sec:how_could_people_be_ends} % (fold)

Kant's Formula of Humanity reads: 
\begin{quote}
	So act that you use humanity, whether in your own person or that of another, always at the same time as an end, never merely as a means. (\textsc{g} 4:429; 4:436) 
\end{quote}

The Formula of Humanity is intended as a general characterization of impermissible action. More than that, it purports to provide a substantive explanation of impermissible action in terms of a distinctive kind of value that people enjoy. Part of the power of the Formula of Humanity consists in this explanatory ambition. Indeed the descriptive and explanatory ambitions are linked. The Formula of Humanity is not a general characterization of impermissible action in the sense that it provides a mechanical algorithm from which more specific duties may be derived. Rather, the Formula of Humanity is a general characterization of impermissible action in the sense that it specifies the substantive value that grounds our duty.

The Formula of Humanity resonates with many people, not all of them Kantians. Yet despite the emotional power of the formula, its content can prove elusive. Let me begin by raising some questions concerning it, not all of which I can hope to answer fully here.

One obstacle to interpretation, as Raz observes, is unclarity about how people could be ends, let alone ends in themselves: 
\begin{quote}
	I can make it my end to get people jobs, or to see it that they come to no harm, or to insure them a comfortable income, or to keep them from temptation, and so on. But can they themselves, rather than securing them something, be my end? \citep[144]{Raz:2001ps} 
\end{quote}
Raz's doubt that people could be ends is not uncommon; indeed, it echoes a doubt earlier raised by Sidgwick:
\begin{quote}
    The conception of `humanity as an end in itself' is perplexing: because by an End we commonly mean something to be realised, whereas `humanity' is, as Kant says, `a self-subsistent' end. \citep[39]{Sidgwick:1981jk}
\end{quote}
To get a person a job is to bring about a certain state of affairs---that the person is employed. And so it is with all of Raz's other examples---each is a potential if nonexistent state of affairs that a person intends to bring about by action. However, while people may figure in states of affairs, people are not themselves states of affairs. So it is hard to understand how people could be ends, let alone ends in themselves, if an end is a state of affairs intentionally brought about by action. This is a doubt endorsed even by those more sympathetic to Kant's ethics than either Sidgwick and Raz \citep[see, for example,][]{Lo:1987xy,Ross:1954jk,Wolff:1973qv}.

Unclarity about how people could be ends might encourage an interpretation of the formula that focuses on its second half---the injunction to treat people never merely as a means.

A natural way of developing this idea is as follows: If there are limitations on how we deal with people not derivable from their potential use for our own ends, then people must have noninstrumental value---people must possess an intrinsic, noninstrumental value that grounds what reasons there are for limiting our treatment of them. And if these limitations may not be overridden, then it can seem that we have captured much of what Kant meant by describing people as ends in themselves. Nozick's notion of a side constraint offers just such interpretation:
\begin{quote}
    Side constraints upon action reflect the underlying Kantian principle that individuals are ends and not merely means \ldots\ There is no side constraint on how we may use a tool, other than the moral constraints on how we may use it upon others \ldots\ there is no limit on what we may do to it to best achieve our goals. Now imagine that there was an overridable constraint C on some tool's use. For example, the tool might have been lent to you only on the condition that C not be violated unless the gain from doing so was above a certain specified amount, or unless it was necessary to achieve a certain specified goal. Here the object is not \emph{completely} your tool, for use according to your wish or whim. But it is a tool nevertheless, even with regard to the overridable constraint \ldots\ If we add constraints on its use that may not be overridden, then the objects may not be used as a tool \emph{in those ways}. \emph{In those respects}, it is not a tool at all. Can one add enough constraints so that an object cannot be used as a tool at all, in \emph{any} respect? \citep[30--1]{Nozick:1974as}
\end{quote}
% TODO Find page references

Raz makes three important criticisms of this interpretation.

First, from the fact that there are limitations on how we deal with people not derivable from their potential use for our own ends, it follows that they have intrinsic, noninstrumental value. It does not, however, follow that they are ends in themselves. Purposeless actions, such as expressive actions, are Raz's example. If it is intelligible in some circumstance for me to kick the table in exasperation, then kicking the table must have some value not derivable from my purpose in kicking the table for it has none, but there is no plausible interpretation of ends in themselves in which such actions qualify.

Second, the idea that there are limitations on our treatment of people that may not be overridden is controversial, and it is natural to want an explanation of this feature rather than merely taking it for granted.

This leads to Raz's third, and most fundamental, criticism. The present interpretation purports to explain the nature of ends in themselves in terms of the permissible forms of treatment of people. But this is to abandon the explanatory ambition of the Formula of Humanity. The notion of an end in itself is supposed to be the \emph{grounds} of such treatment in the sense that in explains and renders intelligible that certain forms of treatment are permissible or impermissible. In order to explain and render intelligible the permissibility or impermissibility of forms of treatment, the nature of ends in themselves must be specified, at least in part, independently of such treatment.

To Raz's three criticisms let me add a fourth. Not only has the explanatory ambition of the Formula of Humanity been abandoned, but so has its descriptive ambition. The formula is meant to be a general characterization of permissible and impermissible treatment of people. It is a general characterization in the sense that it specifies the substantive value that explains and renders intelligible that certain forms of treatment of people are permissible or impermissible. But the present interpretation takes the permissible and impermissible treatment of people as antecedently understood and explains the nature of ends in themselves in terms of such treatment. It accomplishes none of what Kant hoped to achieve.

This is not the only possible interpretation of the injunction to never treat people merely as a means. Perhaps we can do better. \cite{Scanlon:2004ac} observes that one obstacle to such an interpretation is that treating people merely as a means seems to characterize a particular class of wrongs. `You were just using me!' seems to be a charge with a particular moral force not appropriate to all forms of wrongdoing. If someone is drowning and you walk buy without pulling a lever which would summon the emergency services, you have wronged that person, to be sure, but you have not used that person, at least not in the ordinary sense. Again, the descriptive ambition of the Formula of Humanity has been frustrated, at least in its full generality. 

An important observation about the injunction never to treat people merely as means is the essential role of the qualifier `merely'. As H.J. \citet{Paton:1946dz} long ago observed, the Formula of Humanity does not forbid our using another person as a means. When you buy a stamp from a postal clerk you are using the postal clerk as a means for acquiring that stamp despite the clerk's evident humanity, but you have done no wrong. Paton's observation generalizes---without the qualification the Formula of Humanity would forbid all cooperative activity, but that is an implausible and unintended consequence.

This observation is the basis of a fundamental difficulty for any interpretation of the Formula of Humanity that seeks to escape the unclarity of conceiving of people as ends by focusing on the second half of the formula. When you buy a stamp from a postal clerk you are using the postal clerk as a means for acquiring that stamp despite the clerk's evident humanity, but you have done no wrong. Rather, what is forbidden is omitting to treat a person `at the same time as an end.' Thus the injunction to treat people `never merely as a means' is redundant: The moral content of the Formula of Humanity consists \emph{entirely} in the injunction to treat people as ends in themselves \citep[for further criticism of attempts to derive moral content from the prohibition to treat people merely as a means see][]{Parfit:2008lr}. So the unclarity that Raz finds in conceiving people as ends must be squarely faced. (There remains a question, however: If the moral content of the Formula of Humanity consists entirely in the injunction to treat people as ends in themselves, then why emphasize that one should never treat people merely as a means? An answer will emerge in \S \ref{sub:on_not_treating_people_merely_as_a_means})

\subsection{Ends} \label{sub:ends} % (fold)

I believe that this unclarity is due to an overly narrow construal of an end and that Kant is guilty of no such conflation. % TODO Give page reference to Raz's charge of conflation 

It is natural and appropriate to speak of a state of affairs intentionally brought about by action as the end of that action. So in writing the candidate a reference, having an informal word with potential employers, and so on, the end of my action is a state of affairs---that the person is employed. However, this does not entail that \emph{all} ends are states of affairs intentionally brought about by action nor is this latter claim true. A state of affairs is the end of a person's action insofar as that person acts for the sake of bringing about that state of affairs. So in writing a candidate a reference, having an informal word with potential employers, and so on, one performs these actions \emph{for the sake of} bringing about a certain state of affairs, that the person is employed. Fundamentally, ends are that for the sake of which we act (or refrain from acting) \citep[see, inter alia,][]{Velleman:2006nx}. Sometimes we act for the sake of bringing about some state of affairs, but this is not inevitable. We can, for example, act for the sake of preserving our continued existence and well-being, but our continued existence and well-being, if these are things genuinely to be \emph{preserved}, already exist. Thankfully, in normal circumstances, at least for academics in the West, these are things which we need not actively pursue. Nevertheless, even in such sanguine circumstances, they set limits on the ends which we can reasonably adopt \citep[see][88-94]{Korsgaard:1996md}. 

Can people be ends in this broader and more fundamental sense? Of course. A parent can act for the sake of his child by providing for the child's education. In providing for the child's education, the parent acts for the sake of the child. The child is, in this broader sense, the end of the parent's action. In general, to act for the sake of something is to suppose that thing has a value that grounds a reason that renders intelligible your so acting.

% subsection ends (end)

\subsection{Ends in Themselves} \label{sub:ends_in_themselves} % (fold)

What kind of value do people manifest that makes them ends in themselves? Kant entertains the following answer: 
\begin{quote}
	But suppose there were something the existence of which in itself has an absolute worth, something which as an end in itself could be a ground of determinate laws; then in it, and in it alone, would lie the ground of a possible categorical imperative, that is, of a practical law. (\textsc{g} 4:428) 
\end{quote}
Kant is supposing the existence of something which is \emph{an end in itself}, \emph{an existent end}, and has \emph{absolute worth}. 

Let's consider these in turn.

First, Kant is supposing the existence of something that is an end in itself. An end in itself is something whose value is unconditional, independent of inclination, and valid for all rational beings. Kant contrasts ends in themselves with relative ends, ends whose value is conditional, dependent on inclination, and varies from one rational being to another. Relative ends could only the grounds of hypothetical imperatives (\textsc{g} 4:428). Conditional values are not good unconditionally but are only good under certain conditions; under different conditions, they can be without value or even bad. Even secondary virtues such as moderation, self-control, and calm reflection, can be extremely evil if used by a bad will (as when the coolness of a scoundrel not only makes him more dangerous but also more \emph{abominable}). So the value of the conditionally valuable depends on the value of what conditions it. Kant, like Raz, maintains that conditional values are ultimately conditioned by unconditional values. The intuition is that value is \emph{grounded}---the value of things are grounded in what is unconditionally valuable if they are of value at all. 

Second, Kant is supposing the existence of an existent end (\textsc{g} 4:437). An existent end is something that already exists and `whose existence is in itself an end' (\textsc{g} 4:428). Kant contrasts existent ends with ends to be effected. An end to be effected is something that does not yet exist but can be brought about by action (\textsc{g} 4:437). Notice that this contrast only makes sense if ends are are understood in the broader and more fundamental sense of being that for the sake of which we act. Existent ends must be things whose value grounds reasons that make it intelligible that we treat them in certain ways.

Third, Kant is supposing the existence of something that has absolute worth. The contrast might be with things with relative worth. So understood things with absolute value would be ends in themselves. I believe, however, that Kant means more than this. I believe that Kant is anticipating his deployment of the Stoic distinction between dignity and price. (Compare Seneca's distinction between \emph{dignitas} and \emph{pretium}.) Specifically, Kant is supposing the existence of something with dignity. The value of dignity is incomparable; it is a value that cannot be measured against the value of anything else (\textsc{g} 4:434). An end with only relative worth, or price, can be measured against the value of something else, and it would be rational to trade it for something else of equivalent or greater value. It is in this sense that Kant claims that absolute worth is `without limit'. It may be unclear in what sense the value of something could be incomparably greater (since if it is \emph{greater} there is at least one sense in which it is \emph{comparable}). Nevertheless, the idea is intelligible and familiar outside of philosophy, say, when we refuse to sell at any price. It is this feature of dignity, its incomparable value, that explains Kant's identification of respect for the moral law with respect for people considered as ends in themselves. Regarding the value of humanity as an object of comparison dishonors humanity. Proper respect for humanity requires that one only compare one's conduct with the moral law and not with the conduct of other rational beings.

These three features, being an end in itself, being an existent end, and having absolute worth or dignity, are conceptually independent. Thus, for example, nonrational animals and features of the natural environment are existent ends---already existing things for the sake of which we act---but are plausibly not ends in themselves. Similarly, suppose the Messiah would only come if certain conditions were achieved, the coming of the Messiah might be an end in itself and an end to be effected, and so an end in itself need not be an existent end. 

% TODO Find page reference
\citet{Wood:1999zy} observes that one aspect of their conceptual independence highlights the important role of dignity in Kant's account. Notice that it is conceivable that an end in itself might only have relative worth or price. This might seem inconsistent at first, but consider the following: Suppose that there was something whose value was unconditional, independent of inclination, and valid for all rational beings. But suppose as well that the degree of this value was limited. It might be rational to trade it for something with a greater degree of inclination-dependent value. This possibility is raised by Raz's \citeyearpar{Raz:2001ps} observation that the unconditional nature of a value is independent of the stringency of the reasons it gives rise to. Kant implicitly recognizes this as well---the only thing that prevents this possibility in Kant's system is his contention that humanity, in its self-legislative aspect, has dignity or absolute worth and hence cannot be exchanged for a mere price. So the incomparable value of dignity not only explains Kant's identification of respect for the moral law with respect for people considered as ends in themselves, but it also explains the overriding character of moral reasons since the value that grounds competing nonmoral reasons could only be a price.

Despite their conceptual independence, Kant claims that only one thing has all three features---humanity or rational nature. That these features are united in humanity is thus a synthetic claim.

% subsection ends_in_themselves (end)

\subsection{The Derivation} \label{sub:the_derivation} % (fold)

Supposing there is such a thing as an end in itself, what could it be? 

While Kant cannot demonstrate that an end in itself must be humanity or rational nature, he argues that we necessarily, if implicitly, presuppose this: 
\begin{quote}
	The ground of this principle is: rational nature exists as an end in itself. The human being necessarily represents his own existence in this way; so far it is thus a subjective principle of human actions. But every other rational being also represents his existence in this way consequent on just the same rational ground that also holds for me; thus it is at the same time an objective principle from which, as a supreme practical ground, it must be possible to derive all laws of the will. The practical imperative will therefore be the following: So act that you use humanity, whether in your own person or in the person of any other, always at the same time as an end, never merely as a means. (\textsc{g} 4:428-9) 
\end{quote}
The argument here is terse and difficult to interpret. The overall structure of the argument is relatively clear; the main difficulty is in interpreting the first step. It cannot be read as a contingent empirical claim about how people explicitly value their own existence for so interpreted it would be false---unfortunately, some people, in the grips of despair, regard their existence as worthless. So how are we to understand Kant's claim that human beings necessarily represent their own existence as an end in itself? 

According to Kant, our humanity or rational nature consists, at least in part, in our capacity to set ends (\textsc{g} 4:437). Given this, Kant might be understood as claiming that when we value the ends we set for ourselves, we must, implicitly at least, value our capacity to set these ends whether or not we explicitly value this capacity in our own person. 

One influential way of developing this idea is due to \citet{Korsgaard:1996md}. Korsgaard understands Kant's reasoning here as a `regress on conditions'. We begin by observing the value we place on the ends we set and then infer that this value is conferred on these ends by the rational nature that sets them. Korsgaard's idea is, roughly, that being the object of rational choice is the \emph{source} of a thing's value. It is because rational nature is the source of all goodness that it must be the fundamental object of respect since if it is not respected as good, then nothing else can be valued as good.

I won't reconstruct Korsgaard's interpretation, in part, because I do not fully understand it \citep[though, for an elaboration, see][]{Wood:1999zy}. More importantly, however, if Kant is moved by the `value-conferring status' of rational nature, then so much the worse for the derivation, for the operative thought is simply false.

A person, in rationally adopting an end, must, implicitly at least, regard that end as valuable, at least to some extent. (Perhaps only in the minimal sense that not acting on that end would be more trouble than it is worth.) Indeed, the perceived value of the end grounds the accepted reason that would render intelligible acting on it, but it does not follow that the value of the end, if genuine, exists just because one has adopted that end. Not only does the perceived value of the end ground the person's reason to promote it, but it grounds what reason there was to adopt that end in the first place. The value of the end must be antecedent to its adoption. But that means that no sense can be attached to the adoption of an end conferring value upon it \citep[see][]{Scanlon:2004ac}.

Similar remarks apply to the intuitive contrast that motivates Kant's discussion of autonomy. Consider two ways in which a person might obey a law. First, a person might obey a law in order to avoid the sanctions imposed by some external authority---the state or God, say. The person's reason for obedience would be to avoid externally imposed sanctions that he fears. If these sanctions ceased to be enforced, or if he ceased to fear them, then the person would no longer have a reason to obey the law. However, a person might obey the law, not because he is {\itshape subject} to it, but because he is the {\itshape author} of it. Even if no sanctions were enforced, the person, as the author of the law, would still have a reason to obey the law. In legislating a law, a person is bound to it, not by any external incentive, but by the reasons he recognized in legislating it. This, however, could only be so if these reasons exist prior to the legislation. (Compare Raz's \citeyear{Raz:2001kx} claim that the sense of active that marks the boundary between our lives as we lead them and what happens in our lives is best understood in terms of our responsiveness to reasons.)

Perhaps, the claim that the value of an end and the reasons it gives rise to exist prior to its adoption is too extreme. \citet{Raz:2001ps} distinguishes the value of a thing and its value for a person. My setting an end may not confer value on that end, but it might confer its value for me. If the end I set is genuinely worthwhile, then my life goes better if I achieve it. Success or failure in achieving me ends (at least those that are genuinely worthwhile) determines a standard by which my life goes better or worse and affects in obvious ways the meaning or significance of my life. Moreover, insofar as the well-being of person is a reason for others to promote or at least not interfere with it, then my adopting a valuable end, gives rise to a reason that did not exist prior to my adoption of it. However, there are limits to the extent to which my setting an end confers its value for me. Achieving an end is an element of a person's well-being only if that end is genuinely worthwhile. So an end's value for a person depends, in part, on the value of that end. Attributing to rational nature a value-conferring status plausibly conflates an end's value for a person for the value of that end.

Is there a way to interpret the first step of the derivation without invoking the value-conferring status of rational nature? 

The problem is to understand how we can move from the value of the adopted end to the value of the capacity to adopt that end. Korsgaard's idea was that being the object of rational choice is the \emph{source} of a thing's value. We have seen reason to doubt this idea. It might be that something is good only if it can be the object of rational choice, but it does not follow that that thing is made good by being the object of rational choice, that rational nature is the source of its value. An idea of Raz's helps to see how this might be so. To understand this, let's pause and consider Raz's metaphysics of value.

% subsection the_derivation (end)

\subsection{Raz and the Metaphysics of Value}\label{sub:raz_and_the_metaphysics_of_value} % (fold)

% TODO decide whether to include the following passage:

\citet[ch.\ 4, \S 4]{Raz:2001ps} distinguishes two abstract categories of value. Everything that is of value is either:
      \begin{enumerate}
          \item valuable for something or someone else, or
          \item valuable in itself.
      \end{enumerate}
The distinction is between what is conditionally and what is unconditionally valuable: If something is merely valuable for something or someone else, then its value is conditional or dependent on what it is good for, provided that what it is good for is itself good or valuable. If a thing is valuable for something but is not valuable in itself, then it is valuable only insofar as it is good for something which is itself of value. In contrast, if something is valuable in itself then its value is unconditional; its value does not depend on being good for something or someone else. Though, of course, being valuable in itself is consistent with being, at the same time, good for something or someone else---it is just that its being valuable in itself does not depend on its being, at the same time, good for anything else.

An end in itself is something whose value is unconditional and so is valuable in itself. But there is more to the Kantian conception of an end in itself than this abstract category of value---which is why Raz proposes to replace former with the latter given the alleged obscurity of conceiving of people as ends.

This distinction is not the distinction between instrumental and intrinsic value. To be sure, the value of what is instrumentally valuable is conditional on the value of its consequences or likely consequences. But many intrinsic values can be conditionally valuable as well:
\begin{quote}
  When one says that reading Proust enriches one's life one is not pointing to the consequences of the reading. Rather reading Proust with understanding is such enrichment. Engaging with intrinsic goods in the right way and with the right spirit \ldots is good for one in and of itself, in one of many possible ways. \citep[148]{Raz:2001ps}.
\end{quote}
This is another respect in which Raz and Kant differ---Kant makes no allowances for intrinsic, noninstrumental value that is not valuable in itself. (See \ref{sec:conclusion} for an important consequence of this.)

Raz, like Kant, maintains that that valuers are valuable in themselves. To argue for this, he proposes that the following two conditions are jointly sufficient for something to be valuable in itself \citep[151--2]{Raz:2001ps}:
\begin{itemize}
	\item there are things that are good for it;
	\item their being good for it is not conditional on it contributing to the good of something else. 
\end{itemize}
Thus, to establish that that valuers are valuable in themselves, it suffices to argue that there are things that are good for valuers and their being good for valuers is not conditional on valuers contributing to the good of something else. 

Raz clearly argues for the second of the jointly sufficient conditions---that being good for valuers is not conditional on valuers contributing to the good of something else:
\begin{quote}
	To show that valuers not only play the role of being of value in themselves in relation to goods which are good for others, but that they actually are of value in themselves one has, secondly, to show that their good does not matter simply because it is good for someone or something else. \citep[156]{Raz:2001ps}
\end{quote}

The argument for the first of the jointly sufficient conditions---that there are things that are good for valuers---is indirect. The bulk of that discussion is devoted to establishing value's immanence: Something is good only if it can be the object of rational choice in the sense that it is only by being the object of rational choice that its value is \emph{realized}. Novels are to be read, music is to be listened to. It is not that unread novels are only of potential value (the way they would be if being the object of rational choice were the source of their value), it is rather that in going unread their value is wasted. The existence of a value does not depend on being the object of rational choice but its realization does. Given the doctrine of value's immanence, the argument that there are things that are good for valuers proceeds as follows:
\begin{quote}
	If there are intrinsic values whose realisation requires recognition, a recognition which being valuers they can give, then there are things, which are, assuming that the valuers are good, good for them. \citep[156]{Raz:2001ps}
\end{quote}

So, there are things which are good for valuers, and their being good for valuers does not depend on valuers being good for something else, and, hence, valuers are valuable in themselves.

% subsection respect_and_the_metaphysics_of_value (end)

\subsection{A Speculative Reconstruction}\label{sub:a_speculative_reconstruction} % (fold)

Raz's doctrine of value's immanence is the basis of, if not an altogether faithful interpretation of the derivation, then a speculative reconstruction of it that is of some independent interest. Value is immanent: In reading a novel, at least with understanding and appreciation, its value is realized. Not only is the value of the novel realized, but its realization is also good for the person who reads it. So a person's capacity to set a valuable end is a condition not only for the the realization of its value but for the good it affords the person who acts on it. Perhaps, then, in implicitly valuing the ends he sets, a person is committed to valuing his capacity to set them, since that capacity is a condition not only for their realization but the good they afford him. Not only are we implicitly committed to valuing our capacity to set ends, but so is every other rational being `on just the same ground'. So we must understand the value of rational nature as an objective value, valid for all rational beings. If the value of rational nature does not depend on the value of anything else, then this is well explained: Rational nature has a value valid for all rational beings because the capacity to set an end is the fundamental condition for the realization of all value and, as such, is a worthy object of respect: 
\begin{quote}
	But the lawgiving itself, which determines all worth, must for this very reason have a dignity, that is, an unconditional, incomparable worth; and the word respect alone provides a becoming expression for the estimate of it that a rational being must give. (\textsc{g} 4:436) 
\end{quote}
Being the fundamental condition for the realization of all value explains, at least in part, the incomparable value of human dignity. (I say \emph{at least in part} because the full account of what grounds the dignity of humanity involves our autonomous nature as self-legislative members of the kingdom of ends and is ultimately rooted in our noumenal natures.) All rational beings are ends in themselves and this makes an objective claim on all rational beings to recognize themselves and others as ends in themselves. This leads to the formulation of the categorical imperative as the requirement to use humanity in every rational being always as an end and never only as a means.

More would need to be said to fully develop this reasoning, but when properly developed it approaches Raz's argument of chapter four of \emph{Value, Respect, and Attachment}. 

% subsection a_speculative_reconstruction (end)

\subsection{On Not Treating People Merely as a Means} \label{sub:on_not_treating_people_merely_as_a_means} % (fold)

A final issue remains about ends in themselves as Kant conceives of them. If the moral content of the Formula of Humanity consists entirely in the injunction to treat people as ends in themselves, then why emphasize that one should never treat people merely as a means? As we will see, an answer will have consequences for how Kant conceives of the attitude of respect and how Raz's account of this attitude might be refined and elaborated.

The second half of the Formula of Humanity is not a general \emph{characterization} of wrongdoing---Scanlon's doubt on this score is vindicated at least to that extent. Rather, it purports to provide a general \emph{diagnosis} of human wrongdoing. Human beings have a tendency to act contrary to duty, not because they altogether fail to value humanity, but because they misconceive the kind of value it has. Thus, as human beings, given our unsocial sociability, we have a tendency to place greater value on the objects of inclination, which merely have a price or relative worth, over the value of humanity, which has instead a dignity or absolute worth. In exchanging the dignity of humanity for a mere price, we treat humanity merely as a means.

This both complements and illuminates the sense in which the Formula of Universal Law (`Act only in accordance with that maxim through which you can at the same time will that it become a universal law'. \textsc{g} 4:421; 4:402) is meant to be a moral `compass' (\textsc{g} 4:404): 
\begin{quote}
	If we now attend to ourselves in any transgression of duty, we find that we do not really will that our maxim should become a universal law, since that is impossible for us, but that the opposite of our maxim should instead remain a universal law, only we take the liberty of making an exception to it for ourselves (or just for this once) to the advantage of our inclination. Consequently, if we weighed all cases from one and the same point of view, namely that of reason, we would find a contradiction in our own will, namely that a certain principle be objectively necessary as a universal law and yet subjectively not hold universally but allow exceptions. (\textsc{g} 4:424) 
\end{quote}
Thus, when tempted to make false promises when in need of money, we do not will that everyone should make false promises when in need of money. Rather if we succumb to temptation, we will that in general people truly promise only that we, just this once, falsely promise to repay a creditor. So in acting contrary to duty, a person gives unjustifiable preference to his own needs and inclinations over the needs and inclinations of other finite rational beings. The first formula is a moral compass for those lost to temptation, not in the sense that it provides a mechanical algorithm for deriving more specific duties, but as a vivid reminder not to give unjustified preference to one's own needs and inclinations over the needs and inclinations of other finite rational beings, and hence as a corrective to our human propensity to make exceptions for ourselves to universal moral principles. 

Notice that, when we make an exception for ourselves, we fail to express proper respect for the dignity of humanity in ourselves or in the person of another. In making an exception for ourselves, we sacrifice the dignity of humanity for a mere price and, hence, fail to properly respect that dignity. Indeed, Kant anticipates this explicit connection when in the first section of the \emph{Groundwork} he observes that human beings possess: 
\begin{quote}
	\ldots a propensity to rationalize against those strict laws of duty and to cast doubt upon their validity, or at least upon their purity and strictness, and where possible, to make them better suited to our wishes and inclinations, that is, to corrupt them at their basis and to destroy all their \emph{dignity} [my emphasis]---something that even common rational cognition cannot, in the end, call good. (\textsc{g} 4:405) 
\end{quote}
Thus a tendency to misconceive the value of humanity explains a propensity to make exceptions for ourselves to universal moral principles to the advantage of inclination \citep[for a similar account of the role of dignity in explaining the injunction to never treat a person merely as a means see][]{Thomas-Hill:1992nr}. 

How might we resist temptation and avoid corruption?

% TODO Insert reference to Groundwork
We all more or less know right from wrong even if we are tempted to make exceptions for ourselves. Specifically, we are equipped with common rational moral cognition---pre-reflective, practical knowledge of the moral law. Nevertheless, philosophical inquiry is required to gain reflective, theoretical knowledge of the moral law. Just as wisdom has need of science not to learn from it but to gain durability and access to its principles, common rational moral cognition has need of philosophy not to learn from it but to gain durability and access to its principles. This is a manifestation of Kant's Pietist background. Kant is looking for a reasonable form of moral reflection to check the purity of our motives. This form of moral reflection is practically necessary since, without it, we are easily tempted to act from wrong reasons. Thus common rational moral cognition can recognize a practical motive to engage in the philosophical reflection that Kant pursues in the second section of the \emph{Groundwork}. There, Kant argues that whereas the first formula is best for the appraisal of action, for access to the moral law the three formulas should be applied to one and the same action thereby bringing the moral law closer to intuition and thereby feeling.

While it is not yet clear what Kant means by `access to the moral law', Kant is at least claiming this much: With the first formula as an external aid, a person has sufficient means to act in conformity with duty. But by gaining access to the moral law thereby bringing it closer to intuition and thereby feeling, a person forms a strong desire to act from duty even though there is no positive inclination to do so or a strong inclination to act contrary to duty. In forming such a desire, common rational moral cognition ensures the durability of its principle.

The durability of a principle is a matter of the degree and stability of its causal influence. The increased durability of the moral law in a person's conduct is explained as follows: In intuiting the dignity of humanity, the value of objects of inclination pale in comparison if indeed they are experienced to be of value at all. The intuition of human dignity gives rise to a distinctive moral pleasure, the feeling of respect. Together they constitute the pure desire to act from duty since a desire just is the representation of an object accompanied by a feeling of pleasure. What's represented in intuition is the grounds of the reasons for acting from duty, and the feeling of respect is a conscious manifestation of the value of the intuited dignity. It is not possible to respect the dignity of humanity and to take pleasure in the fulfillment of an inclination contrary to duty---the feeling of respect precludes such pleasures. This is a way in which the incomparable value of dignity is consciously manifest to the person who intuits it. From this perspective, the perspective of a virtuous sensibility, inclinations contrary to duty do not seem to be reasons and so are motivationally inefficacious. Moreover, since the pure desire does not depend on inclination for its object, it can operate even in the absence of positive inclination. The pure desire to act from duty thus plays an important explanatory role in the motivational psychology of a virtuous sensibility, even though it is a response to reasons and not their source.

In acting from duty, a virtuous person does so out of a phenomenologically vivid sense of the value that grounds the reasons for so acting. It is this phenomenologically vivid sense of the value of humanity that Kant experiences in reading Rousseau (and that presumably `wrenches' the philanthropist from his `wretched insensibility' \textsc{g} 4:398). Kant has not altogether abandoned the moral sentimentalism of his youth. Indeed, his remarks about the practical interest in the moral law can be understood as a (partial) interpretation of Butler's \citeyearpar{Butler:1736id} gloss on moral sense: Insofar as the intuition of a virtuous sensibility takes as its object the value that grounds our duty it is a `perception of the heart'; insofar as the feeling of respect is part of the noncognitive response to the reasons grounded in this value---insofar as it is `self-wrought from reason' (\textsc{g} 4:401n)---it is a `sentiment of the understanding'. (Compare \citealp{Wiggins:1987ta}, laudable development of Humean sentimentalism.)

Kant's remarks about the practical interest in the moral law is, at best, a \emph{partial} interpretation of Butler's gloss on moral sense in that one important aspect of the metaphor, `perception of the heart', is so far left out of account. That the intuition of a virtuous sensibility takes as its object the value that grounds our duty may vindicate its characterization as a perception---like perception proper it is a mode of awareness. But nothing of Kant's account so far vindicates its characterization as a perception \emph{of the heart}. How might the latter half of Butler's metaphor be understood? One natural interpretation understands the metaphor as the claim that the mode of awareness of the value is only possible if one possesses the appropriate affective sensibility. The thought is that the value could not be the object of our awareness unless we were capable of the appropriate affective response---in Kant's terms, unless we could take pleasure in that value. (Again compare \citealp{Wiggins:1987ta}, development of Humean sentimentalism.) As I have observed, nothing in Kant's account as presented so far justifies an attribution of this claim to Kant. However, it is not inconsistent with what \emph{has} been attributed to Kant. Moreover, the claim \emph{would} explain the sense in which the pleasure taken in the awareness of value is a conscious manifestation of that value. The \emph{feeling} of respect would be no mere effect of our awareness of the dignity that grounds our duty, it would be partly constitutive of that awareness in the sense that it is part of what makes our awareness of that dignity possible. 

How does this extension of Kant's conception of respect compare to Raz's conception of that attitude?

According to \citet[160]{Raz:2001ps}, `respect in general is a species of recognising and being disposed to respond to value, and thereby reason.' Recognizing and being disposed to respond to value consists, on the one hand, having thoughts, imaginings, intentions, emotions, consistent with the recognized value and, on the other hand, recognizing the reasons to not destroy, and perhaps even to preserve, what is of value where such reasons are categorical, at least in the sense of being independent of inclination, if not in the sense of providing overriding reasons. Respect is a response to values and thereby reason, not just in the sense that respect involves the acknowledgement of what reasons there are to not destroy and perhaps to preserve what is of value, but also, more fundamentally, in the sense that the recognized value grounds reasons. That it makes sense for Edgar to throw Bernice a party might be determined by the value of doing so---that it would be a wonderful surprise. So understood values ground reasons that render intelligible people's attitudes. 

As normally understood, recognition is, or at least involves, a mode of awareness. So if respect is a species of recognition of value, then respect at least involves an awareness of value. So understood, Raz's account of respect is thus far consistent with the role of intuition in Kant's account. While the language of recognition may license this interpretation, I confess it is unclear to me whether respect for Raz involves judgment rather than awareness. A more definite difference concerns the noncognitive elements in the response to value. Affect \emph{does} play a role in Raz's account---part of what it is to respect a person is to have emotional attitudes consistent with their value: `despising someone as worthless, or mean, when he is in fact generous and kind, is having an emotion inconsistent with his value' \citep[161]{Raz:2001ps}. However, this claim is limited in at least two respects. First, it is merely the negative claim that one should not have emotional attitudes inconsistent with the recognized value. But this is distinct from the further positive claim that one should have the emotional attitudes appropriate to the recognized value such as Kant's affective response to the dignity of humanity disclosed by Rousseau's writing. Second, these noncognitive elements are understood as a mere effects of the awareness of value rather than being understood as playing a constitutive role in making such awareness possible. 

Since, for Raz, values give rise to reasons, an emotional attitudes elicited by awareness of value is a response to reasons and is in that sense a sentiment of the understanding. Like Kant, respect for Raz involves a mode of awareness of value and in that sense is a perception. Like Kant, there's no evidence that Raz regards recognition of value as the perception of the heart, but such a claim is a consistent extension of Raz's account. That affect does not play a larger role in Raz's account and that he is insensitive to the explantory role that it may play is due in large part to his skepticism that desires could be reasons. But such skepticism is consistent with the explanatory role that pure desire plays in the motivational psychology of a virtuous sensibility and its potential moral significance. 


% subsection on_not_treating_people_merely_as_a_means (end)

% section how_could_people_be_ends (end)

\section{Conclusion} \label{sec:conclusion} % (fold)

Let me summarize the main points so far: 
\begin{itemize}
	\item Respect is not a feeling, it is a way of treating people. The feeling of respect is part of the noncognitive response to the reasons there are for such treatment. While the feeling of respect is part of the response to these reasons and not their source, it nevertheless plays an important explanatory role in the motivational psychology of a virtuous sensibility. 
	\item What is the object of respect---the moral law or people considered as ends in themselves? Kant is best understood as proposing a substantive identification: Respect for the moral law \emph{just is} respect for people considered as ends in themselves. This is a consequence of the kind of value that people enjoy---the dignity of humanity is incomparable. 
	\item People could be ends only in the fundamental sense of that for the sake of which we act. To act for the sake of something is to suppose that thing has a value that grounds reasons that make intelligible your so acting. What makes people ends and indeed ends in themselves is the dignity of humanity. 
	\item Humanity or rational nature is the fundamental condition for the realization of all value but not by being its source. Being the fundamental condition for the realization of all value explains, at least in part, the incomparable value of human dignity. 
\end{itemize}
On the present interpretation, Kant is no formalist---our duties are grounded, not in the form of practical deliberation, but in a substantive value that grounds the reasons that are the subject matter of practical deliberation. Nor is Kant a constructivist---as rational beings we may be potential authors of the moral law, but in legislating a law we would be bound to it by the reasons we would recognize in so legislating. These reasons must exist prior to our legislation and so could not be constructed by that legislation. 

There are clear ways in which Kant's views can be improved that Raz's discussion brings to light. To take just one example, on Kant's view, what's not and end in itself is valuable only insofar as it is a potential end for a rational being---it merely has relative worth. In effect, Kant makes no allowances for intrinsic, noninstrumental value that is not valuable in itself. This failure is responsible, in part, for Kant implausibly identifying the whole of morality with what \citet{Gibbard:1990bh} describes as `morality in the narrow sense' and what \citet{Scanlon:1998hb} describes as the domain of `what we owe to each other'. Specifically, it leads him to analyze duties to nonrational beings such as animals and features of the natural environment as duties owed to \emph{persons}. But this is implausible. The reason it is impermissible to pave over the Grand Canyon on a whim is not, or not merely, that it would spoil the view for other human beings. To suppose otherwise would be to mistake the value of an awe-inspiring view (the importance for a person of experiencing the Grand Canyon first hand) for the value that makes its view awe-inspiring (its unserveyability, the perilous perspective that its view affords). As Raz argues, instrumental and intrinsic value differ: What makes something instrumentally valuable is its use for a person's ends; what makes something intrinsically valuable are its good-making features. That things are intrinsically valuable provides us with a better understanding of our duties to nonrational beings. They are not duties owed to persons; they are duties owed to valuable things. Moreover, duties to nonrational beings are categorical, at least in the sense of being independent of inclination, if not in the sense of providing overriding reasons.

Value plays different roles in Kant and Raz's accounts of respect. For Kant, the source of disrespect is a failure to properly conceive the value of a person. For Raz, the source of disrespect is a failure to recognize or at least acknowledge the value of a thing. These are distinct, if not incompatible accounts---indeed they may be complementary. Failure to acknowledge value may be one source of disrespect, and misconceiving the value of things may be another. 

It is yet unclear why the value of dignity should necessarily override price. As Raz emphasizes, that there are overriding reasons is a substantive and controversial claim that should be explained if true. Kant's full account of what grounds the dignity of humanity involves our autonomous nature as self-le\-gis\-la\-tive members of the kingdom of ends and is ultimately rooted in our noumenal natures. Constructivists might take heart in this, since a constructivist interpretation of Kant's derivation of the Formula of Humanity might be thought to have the resources to explain and justify the overriding character of dignity. However, there are interpretive and philosophical doubts about any such hope. At the point of the \emph{Groundwork} where the derivation occurs, while the dignity of humanity may be evident, Kant so far lacks an explanation for why humanity enjoys this dignity. Insofar as a constructivist interpretation attributes such an explanation at this point, that is reason to believe it to be a misinterpretation. Moreover, any such explanation is bound to be philosophically inadequate---we may be potential authors of the law, but in legislating a law we would be bound to it by the reasons we would recognize in so legislating. No constructivist interpretation of human dignity could be consistent with this evident observation. 

\begin{table}
	[ht] \setlength{\abovecaptionskip}{0pt} \setlength{\belowcaptionskip}{10pt} \caption{Abbreviations} 
	\begin{tabular*}
		{\textwidth}{lp{0.85 
		\textwidth}}
	    \textsc{ak} & \emph{Immanuel Kants Schriften} Writings of Immanuel Kant\\
		\textsc{g} & \emph{Grundlegung zur Metaphysik der Sitten} Groundwork of the metaphysics of morals 1785\\
		\textsc{kpv} & \emph{Kritik der Praktischen Vernunft} Critique of practical reason 1788\\
		\textsc{ku} & \emph{Kritik der Urteilskraft} Critique of the power of judgment 1790\\
		\textsc{ma} & \emph{Mutmasslicher Anfang der Menschengeschichte} Conjectural beginnings of human history 1786\\
		\textsc{ms} & \emph{Metaphysik der Sitten} Metaphysics of morals 1797--8\\
		\textsc{ve} & \emph{Vorlesungen \"uber Ethik} Lectures on ethics \\
	\end{tabular*}
\end{table}

% (end)

% Bibliography
\bibliographystyle{plainnat} 
\bibliography{Philosophy.bib}

\end{document} 
